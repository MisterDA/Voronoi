\documentclass{article}

\usepackage[utf8]{inputenc}
\usepackage[T1]{fontenc}
\usepackage[french]{babel}
\usepackage{mathtools}
\usepackage{textcomp}
\usepackage{fullpage}
\usepackage{url}
\usepackage{ocamldoc}

\begin{document}

\title{Projet de programmation fonctionnelle\\
Coloriage de cartes\\
Rapport}
\author{Antonin Décimo\\ Anya Zibulski}
\date{\today}
\maketitle

\section{Interface}
Pour jouer, le joueur sélectionne la couleur souhaitée parmi les cases du haut
de la fenêtre. Il peut ensuite colorer les régions du diagramme. Le bouton en
haut à droite montre la couleur selectionnée. Si une coloration valide est
trouvée, un message de félicitations est affiché.

Le diagramme peut être réinitialisé avec le bouton \textit{Nettoyer}, et une
coloration valide (s'il en existe une) est affichée grâce au bouton
\textit{Solution}. Pour quitter, il suffit de cliquer sur le bouton
\textit{Quitter} ou de cliquer sur la touche \textit{Echap.}.

Le premier diagramme de la liste est affiché. L'utilisateur peut se déplacer à
tout moment dans la liste des diagrammes et choisir celui avec lequel jouer.
Le diagramme actuel est alors réinitialisé avant de passer au nouveau.

L'utilisateur peut également changer de fonctions de distance, provoquant
également la réinitialisation du diagramme. Nous proposons quelques fonctions,
accessibles depuis les touches numériques. Un index des distances est affiché en
bas à droite.

La touche \textit{g} permet d'afficher le graphe d'adjacence des régions.

\section{Forme Normale Conjonctive}

Soit $G = (S, A)$ un graphe, $C$ l'ensemble des couleurs ($|C| := 4$). \\
$p_{i, j}$ : la région $i \in S$ est de couleur $j \in C$.
\begin{itemize}
    \item Existence $\phi$ : chaque région reçoit au moins une couleur
    \item Unicité $\varphi$ : chaque région reçoit au plus une couleur
    \item Adjacence $\psi$ : deux régions adjacentes ne reçoivent pas la même couleur
\end{itemize}
Si $F = \phi \land \varphi \land \psi$ est validée par $\bar{v}$ alors $\bar{v}$
fournit une coloration du graphe selon le théorème des quatre couleurs.

$$\phi = \displaystyle\bigwedge_{i \in S}\displaystyle\bigvee_{j \in C} p_{i, j}$$
$$\varphi
  = \displaystyle\bigwedge_{i \in S}
    \bigwedge_{\substack{j, j' \in C \\
    j \neq j'}}
    \neg(p_{i,j} \land p_{i, j'})
  = \displaystyle\bigwedge_{i \in S}
    \bigwedge_{\substack{j, j' \in C \\
    j \neq j'}}
    (\neg p_{i,j} \lor \neg p_{i, j'})$$
$$\psi
  = \bigwedge_{\substack{(i, i') \in A \\
    i \neq i'}}
    \displaystyle\bigwedge_{j \in C}
    \neg(p_{i,j} \land p_{i', j})
  = \bigwedge_{\substack{(i, i') \in A \\
    i \neq i'}}
    \displaystyle\bigwedge_{j \in C}
    (\neg p_{i,j} \lor \neg p_{i', j})
    $$

\input{ocamldoc.tex}
\end{document}
