\documentclass{article}

\usepackage[utf8]{inputenc}
\usepackage[T1]{fontenc}
\usepackage[french]{babel}
\usepackage{mathtools}

\begin{document}
Soit $G = (S, A)$ un graphe, $C$ l'ensemble des couleurs ($|C| := 4$). \\
$p_{i, j}$ : la région $i \in S$ est de couleur $j \in C$.
\begin{itemize}
    \item Existence $\phi$ : chaque région reçoit au moins une couleur
    \item Unicité $\varphi$ : chaque région reçoit au plus une couleur
    \item Adjacence $\psi$ : deux régions adjacentes ne reçoivent pas la même couleur
\end{itemize}
Si $F = \phi \land \varphi \land \psi$ est validée par $\bar{v}$ alors $\bar{v}$
fournit une coloration du graphe selon le théorème des quatre couleurs.

$$\phi = \displaystyle\bigwedge_{i \in S}\displaystyle\bigvee_{j \in C} p_{i, j}$$
$$\varphi
  = \displaystyle\bigwedge_{i \in S}
    \bigwedge_{\substack{j, j' \in C \\
    j \neq j'}}
    \neg(p_{i,j} \land p_{i, j'})
  = \displaystyle\bigwedge_{i \in S}
    \bigwedge_{\substack{j, j' \in C \\
    j \neq j'}}
    (\neg p_{i,j} \lor \neg p_{i, j'})$$
$$\psi
  = \bigwedge_{\substack{(i, i') \in A \\
    i \neq i'}}
    \displaystyle\bigwedge_{j \in C}
    \neg(p_{i,j} \land p_{i', j})
  = \bigwedge_{\substack{(i, i') \in A \\
    i \neq i'}}
    \displaystyle\bigwedge_{j \in C}
    (\neg p_{i,j} \lor \neg p_{i', j})
    $$
\end{document}
